\documentclass{article}
\usepackage[dutch]{babel}

\title{Voorstel Verbetering}
\author{Yujian Jiang, Max Jansen}

\begin{document}
\maketitle

\newpage
\tableofcontents

\newpage
\section{Inleiding}
Voor het project Data-driven Business is het ons helaas niet gelukt om alles op tijd af te ronden. Tijdens de herkansing willen we onze producten verbeteren en afmaken. In dit document kijken we terug op wat er misging en waar er ruimte is voor verbetering.

\newpage
\section{Evaluatie}
Uit onze evaluatie van de feedback blijkt dat er nog veel verbeterd kan worden. We hadden overal wel een begin gemaakt, maar door verschillende omstandigheden was er te weinig tijd om het goed af te ronden.
\\ \\
Hieronder benoemen we per onderdeel waar het misging en wat er beter kan.

\subsection{Projectaanpak}
Binnen onze vorige projectgroep liep de communicatie niet goed. Dit zorgde voor veel verwarring, waardoor het eindproduct niet is geworden wat het had moeten zijn.

\subsection{Pipeline}
Door de tijdsdruk aan het einde van het project hebben we geen werkend model kunnen opleveren. De basis stond er, maar het uiteindelijke resultaat was onvoldoende.

\subsection{Applicatie}
Ook de applicatie was niet af. Door tijdgebrek konden we deze niet op de juiste manier uitwerken, hoewel we wel al een begin hadden gemaakt.

\subsection{Rapport}
Het rapport was net als de rest van het project incompleet. Door de haast en gebrek aan structuur was het niet op het gewenste niveau.

\newpage
\section{Verbetering}
Voor deze tweede poging willen we onze aanpak volledig verbeteren. We gaan beter samenwerken, duidelijke afspraken maken, en het werk beter verdelen. Daarnaast stellen we doelen om ervoor te zorgen dat we wel alles op tijd kunnen afronden.

\newpage
\section{Aanpak}
We starten met een duidelijke planning en taakverdeling. Elke dag houden we een korte meeting om de voortgang te bespreken. Daarnaast zorgen we voor meer structuur in de documentatie, en werken we stap voor stap toe naar een werkend model en een volledige applicatie.

\end{document}
